\chapter{Introdução}

\section{Contextualização}

Para as empresas/organizações se tornarem mais competitivas e possuir melhor desempenho organizacional elas estão adotando a utilização de \textit{frameworks} de arquitetura empresarial, conhecidos como EA - {\textit{Enterprise Architecture}} que faz uso da modelagem conceitual e engloba conceitos estratégicos, estruturais e comportamentais \cite{yu2006exploring}.


\section{Questão de Pesquisa}
A questão de pesquisa para o desenvolvimento deste trabalho é " \textit{Como pode ser realizado a identificação do requisito funcional e não funcional de forma intencional, com intuito de chegar a uma arquitetura pré-definda pela empresa?}"

As diretrizes médicas recomendam organizar as questões de pesquisa segundo três visões: \textbf{População}: Que são as pessoas diretamente afetadas pela intervenção. \textbf{Intervenção}: Que usualmente são uma comparação entre dois ou mais tratamentos alternativos. \textbf{Saídas}: A forma como a intervenção vem sendo aplicada. \cite{keele2007guidelines}. Logo, tomando como base essas diretrizes temos que: 

\begin{itemize}
	\item População: "\textit{..empresas com padrões arquiteturais de software definidos, mas não são/estão implementados de maneira correta..}"
	\item Intervenção: "\textit{..comparação com resultados de estudos de caso existentes na literatura, que utilizam outros modelos ou até mesmo modelos intencionais..}"
	\item Saídas: "\textit{..a forma como é aplicada a modelagem intencional cuja meta é a arquitetura de software pré-definida..}
\end{itemize}
\section{Justificativa}

A utilização da modelagem intencional no contexto de organizações promove aos atores que suas metas sejam mais claras aumentando a transparência relacionada a transformação do negócio da empresa e através dessas transparências das partes interessadas é possível estabelecer uma base sistemática das implicações de design arquitetônico que pode ser colocado. Isso promove que ao analisar diferentes objetivos as decisões a serem tomadas para selecionar diferentes arquiteturas empresariais específicas podem ser realizadas de forma racional. Essa relação das metas com as arquiteturas específicas, o contexto e as atividades de transformação ficam documentados  e podem ser rastreados quando, por exemplo, justificam ações passadas ou revisitam as decisões tomadas em reuniões \cite{yu2006exploring}.

\section{Objetivos}

Este trabalho visa alcançar os objetivos, Geral e Específicos, apresentados a seguir.  

\subsection{Objetivo Geral}

Realizar o desenvolvimento de uma abordagem utilizando modelagem intencional com foco em requisitos não-funcionais para a arquitetura de software pré-definida. 

\subsection{Objetivos Específicos}

Com intuito de alcançar o objetivo geral, são considerados relevantes os seguintes objetivos específicos:

\begin{itemize}
	\item pesquisar modelos intencionais que dão suporte a EA, em nível de arquitetura de dados, arquitetura de tecnologia (capacidades lógicas de software e hardware necessárias para suportar o modelo de negócios);
	
	\item Elaborar um catálogo de operações utilizando o NFR para alcançar um \textit{softgoal} específico de acordo coma a intencionalidade de um cliente simulado;
	
	\item Identificar e selecionar as ferramentas de modelagem intencional que suportam o NFR Framework;
	
\end{itemize}

\chapter{Metodologia}