\chapter{Introdução}

\section{Contextualização}

Os requisitos não-funcionais (NFRs) e funcionais descrevem as características de um software em "como?" ele deve fazer e "o que?" ele deve fazer \cite{sommerville1997requirements}  dentre os dois tipos de requisitos os NFRs na maioria das vezes são declarados de forma enganosa \cite{eckhardt2016non} e são relatados mais de 114 classes de NFRs diferentes, sendo que dentre elas as mais mencionadas na literatura são: 1º Desempenho, 2º Confiabilidade e  3º Usabilidade \cite{mairiza2010investigation}, quando se trata do contexto industrial as classes de NFRs mais utilizadas nas especificações são: 1º Segurança, 2º Confiabilidade e 3º Usabilidade \cite{eckhardt2016non}.

Em termos de desenvolvimento de software os NFRs tem grande impacto quando se trata dos aspectos de implementação, manutenção, operacionalização e utilização de recursos, esses impactos também influenciam os aspectos arquiteturais de uma aplicação de software, quanto maior e mais complexo for o software mais importante os NFRs. Para realizar o desenvolvimento de um sistema de software deve ser considerado em sua arquitetura os aspectos não funcionais \cite{buschmann1996system}. Dentre os padrões arquiteturais um padrão que é universalmente aceito e utilizado em várias linguagens de programação e frameworks e um dos mais utilizados se tratando em tecnologias web existe o padrão \textit{Model-View-Controler} (MVC) \cite{jailia2016behavior}.

O MVC é um padrão arquitetural para sistemas interativos, que divide a aplicação em três componentes a \textit{Model} que contém a implementação das funcionalidades principais e os dados da aplicação, a \textit{View} que exibe as informações da aplicaçãSo ao usuário e a \textit{Control} que lidam com as entradas do usuário \cite{buschmann1996system}. 

\section{Questão de Pesquisa}
A questão de pesquisa para o desenvolvimento deste trabalho é " \textit{Como realizar a modelagem dos NFRs de forma a impactar positivamente as necessidades arquiteturais do MVC de acordo com o atributo de qualidade de segurança?}"

As diretrizes médicas recomendam organizar as questões de pesquisa segundo três visões: \textbf{População}: Que são as pessoas diretamente afetadas pela intervenção. \textbf{Intervenção}: Que usualmente são uma comparação entre dois ou mais tratamentos alternativos. \textbf{Saídas}: A forma como a intervenção vem sendo aplicada. \cite{keele2007guidelines}. Logo, tomando como base essas diretrizes temos que: 

\begin{itemize}
	\item População: "\textit{..Softwares desenvolvidos utilizando arquitetura MVC..}"
	\item Intervenção: "S\textit{..utilização do NFR framework para modelar os requisitos não-funcionais de segurança, realizando a decomposição das características de segurança em operacionalizações  ..}"
	\item Saídas: "\textit{..Um catálogo de operações com as possíveis soluções para serem tratas nas camandas do MVC..}
\end{itemize}
\section{Justificativa}

A utilização da modelagem intencional no contexto de organizações promove aos atores que suas metas sejam mais claras aumentando a transparência relacionada a transformação do negócio da empresa e através dessas transparências das partes interessadas é possível estabelecer uma base sistemática das implicações de design arquitetônico que pode ser colocado. Isso promove que ao analisar diferentes objetivos as decisões a serem tomadas para selecionar diferentes arquiteturas empresariais específicas podem ser realizadas de forma racional. Essa relação das metas com as arquiteturas específicas, o contexto e as atividades de transformação ficam documentados  e podem ser rastreados quando, por exemplo, justificam ações passadas ou revisitam as decisões tomadas em reuniões \cite{yu2006exploring}.

\section{Objetivos}

Este trabalho visa alcançar os objetivos, Geral e Específicos, apresentados a seguir.  

\subsection{Objetivo Geral}

Realizar o desenvolvimento de uma abordagem utilizando o NFR framework para identificar as operacionalizações mais relevantes de acordo com o \textbf{requisito não funcional} para a arquitetura de software MVC \textit{Model-View-Controler}. 

\subsection{Objetivos Específicos}

Com intuito de alcançar o objetivo geral, são considerados relevantes os seguintes objetivos específicos:

\begin{itemize}
	
	\item Elaborar um catálogo de operações utilizando o NFR para alcançar um \textit{softgoal} específico de acordo coma a intencionalidade de um cliente simulado;
	
	\item Identificar e selecionar as ferramentas de modelagem intencional que suportam o NFR Framework;
	
	\item Pesquisar e compreender como é o comportamento e a comunicação entre as camadas de do padrão arquitetural 
\end{itemize}

\chapter{Metodologia}