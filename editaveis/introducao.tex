\chapter{Introdução}

\section{Contextualização}

\section{Questão de Pesquisa}
A questão de pesquisa para o desenvolvimento deste trabalho é " \textit{Como pode ser realizado a identificação do requisito funcional e não funcional de forma intencional, com intuito de chegar a uma arquitetura pré-definda pela empresa?}"

As diretrizes médicas recomendam organizar as questões de pesquisa segundo três visões: \textbf{População}: Que são as pessoas diretamente afetadas pela intervenção. \textbf{Intervenção}: Que usualmente são uma comparação entre dois ou mais tratamentos alternativos. \textbf{Saídas}: A forma como a intervenção vem sendo aplicada. \cite{keele2007guidelines}. Logo, tomando como base essas diretrizes temos que: 

\begin{itemize}
	\item População: "\textit{..empresas com padrões arquiteturais de software definidos, mas não são/estão implementados de maneira correta..}"
	\item Intervenção: "\textit{..comparação com resultados de estudos de caso existentes na literatura, que utilizam outros modelos ou até mesmo modelos intencionais..}"
	\item Saídas: "\textit{..a forma como é aplicada a modelagem intencional cuja meta é a arquitetura de software pré-definida..}
\end{itemize}
\section{Justificativa}

\section{Objetivos}

\subsection{Objetivo Geral}

\subsection{Objetivos Específicos}

