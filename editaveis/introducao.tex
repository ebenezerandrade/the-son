\chapter{Introdução}

Neste capítulo, são descritos o contexto, no qual o trabalho está inserido; a questão de pesquisa, a qual norteia o embasamento científico do trabalho;  a justificativa, visando destacar a contribuição dessa pesquisa; os objetivos geral e específicos, com os quais são acordadas as principais metas a serem atingidas para conclusão do trabalho, e por fim, a organização dos demais capítulos dessa monografia.

\section{Contextualização}

Os requisitos não-funcionais (RNFs) e funcionais descrevem as características de um software em "como esse software deve fazer" e "o que esse software deve fazer". \cite{sommerville1997requirements} dentre os dois tipos de requisitos, os RNFs na maioria das vezes, são especificados de forma enganosa ou ainda são negligenciados, não sendo especificados \cite{eckhardt2016non}  Visando auxiliar os Engenheiros de Requisitos na tarefa de especificar os RNFs, os autores em \cite{mairiza2010investigation} destacam 114 classes de RNFs. Dentre essas classes, as mais mencionadas na literatura são: 1º Desempenho, 2º Confiabilidade e  3º Usabilidade, No contexto empresarial, ou seja mais aplicado e comercial, as classes de RNFs mais utilizadas nas especificações são: 1º Segurança, 2º Confiabilidade e 3º Usabilidade \cite{eckhardt2016non}. Portanto, esse trabalho procura contribuir com a especificação do RNF de segurança.

Ao desenvolver um software, os RNFs impactam, segundo "addreference", na implementação, manutenção, operacionalização e utilização de recursos. Adicionalmente, impactam em aspectos arquiteturais de uma aplicação de software, sendo necessário especificar a arquitetura orientando-se não apenas pelos requisitos funcionais, mas também pelos RNFs \cite{buschmann1996system}

Dependendo do padrão arquitetural utilizado, as especificações bem como as operacionalizações dos RNFs podem variar "addreference". Diante do exposto, e visando focar em um padrão arquitetural comumente utilizado, o presente trabalho pretende orientar-se pelo padrão arquitetural \textit{Model-View-Controller} (MVC). Trata-se de um padrão bem aceito, bem como utilizado no desenvolvimento de aplicações Web, sendo inclusive base para o desenvolvimento dessas aplicações em frameworks e plataformas de geração de código mais emergentes Orientadas a Convenção sobre Configuração \cite{jailia2016behavior}.


O MVC é um padrão arquitetural para sistemas interativos, que divide a aplicação em componentes a \textit{Model}, que contém a implementação das funcionalidades principais e os dados da aplicação; a \textit{View}, que exibe as informações da aplicação ao usuário e a \textit{Controller}, que lida com as entradas do usuário \cite{buschmann1996system}. Os aspectos arquiteturais de um software estão diretamente ligados aos RNFs.



Existem alguns frameworks conceituais centrados na especificação de RNFs. Dentre eles, destacam-se: NFR Framework \cite{chung2009non}, i* (REFERÊNCIA) e FURPS (REFERÊNCIA). Detalhes sobre esses frameworks serão cobertos no Capítulo 2 dessa monografia, Referencial Teórico. Em um primeiro momento, pode-se destacar que os dois primeiros, NFR Framework e i*, fazem uso de uma abordagem mais emergente para especificação desses RNFs. Já o FURPS pode ser entendido como um documento simples, especificado em linguagem natural que se orienta por alguns RNFs comumente encontrados na literatura, no caso (em inglês): \textit{Functionality}, \textit{Usability}, \textit{Reliability}, \textit{Performance} e \textit{Supportability}.
A abordagem mais emergente usa modelos específicos, focados na especificação de requisitos usando princípios da Goal-Oriented Requirements Engineering (GORE) (REFERÊNCIA). Rastreabilidade de requisitos (REFERÊNCIA); especificação de impactos e interdependências entre os requisitos; registro de alternativas quanto aos requisitos analisados, e definição de operacionalizações para viabilizar a realização dos requisitos são algumas vantagens que ficam evidentes no uso desses modelos mais específicos, como, por exemplo: o Softgoal Interdependency Graphs (SIGs) (REFERÊNCIA) ou, em português, Grafos de Interdependências entre Requisitos Não Funcionais. 

Portanto, esse trabalho propõe a definição de um Catálogo de SIGs focado no RNF Segurança, especificamente desenhado para o Padrão Arquitetural MVC, usando a notação do NFR Framework. A intenção é auxiliar os Engenheiros de Requisitos e ainda os Engenheiros de Software na tarefa de especificação de requisitos não funcionais e funcionais quando uma arquitetura é imposta. Como o catálogo ficaria muito abrangente e generalista, caso fosse definido para qualquer RNF e padrão arquitetural, não sendo, portanto, útil, optou-se por focar: (i) no RNF mais preocupante - Segurança, segundo a literatura - em aplicações de software de cunho comercial; (ii) em aplicações Web, e (iii) no padrão arquitetural MVC.  Vale ressaltar ainda que os termos RNFs, atributos de qualidade, critérios de qualidade e metas flexíveis serão utilizados como sinônimos ao longo dessa monografia.


\section{Questão de Pesquisa}
A questão de pesquisa para o desenvolvimento deste trabalho é " \textit{Como realizar a modelagem dos NFRs de forma a impactar positivamente as necessidades arquiteturais do MVC de acordo com o atributo de qualidade de segurança?}"

As diretrizes médicas recomendam organizar as questões de pesquisa segundo três visões: \textbf{População}: Que são as pessoas diretamente afetadas pela intervenção. \textbf{Intervenção}: Que usualmente são uma comparação entre dois ou mais tratamentos alternativos. \textbf{Saídas}: A forma como a intervenção vem sendo aplicada. \cite{keele2007guidelines}. Logo, tomando como base essas diretrizes temos que: 

\begin{itemize}
	\item População: "\textit{..projetos de software utilizando a arquitetura MVC..}"
	\item Intervenção: "\textit{..utilização do NFR framework para modelar os requisitos não-funcionais de segurança, realizando a decomposição das características de segurança em operacionalizações  ..}"
	\item Saídas: "\textit{..Um catálogo de operações com as possíveis soluções para serem tratadas nas camandas do MVC..}
\end{itemize}
\section{Justificativa}

Quando se trata conceitos de arquitetura de software os NFRs são frequentemente chamados de atributos de qualidade e definem um conjunto de preocupações relacionadas aos conceitos de qualidade \cite{schneidewind1990standard} ou classes de NFRs. Os NFRs na maioria das vezes não são tratados de forma correta e as vezes são tratados de forma semelhante aos requisitos funcionais \cite{eckhardt2016non}.

Dado a forma como vem sendo tratados os NFRs este trabalho visa a criação de um catálogo de operações com as possíveis soluções a serem tratadas nas camadas da arquitetura MVC, afim de promover um formato adequado de tratar os NFRs se tratando dos aspectos arquiteturais. O atributo de qualidade que será aplicado neste trabalho é o atributo de segurança, pois é o mais utilizado no meio industrial.  

\section{Objetivos}

Este trabalho visa alcançar os objetivos, Geral e Específicos, apresentados a seguir.  

\subsection{Objetivo Geral}

Realizar o desenvolvimento de uma abordagem utilizando o NFR framework para identificar as operacionalizações mais relevantes de acordo com o \textbf{requisito não funcional} para a arquitetura de software MVC \textit{Model-View-Controler}. 

\subsection{Objetivos Específicos}

Com intuito de alcançar o objetivo geral, são considerados relevantes os seguintes objetivos específicos:

\begin{itemize}
	
	\item Elaborar um catálogo de operações (com tipos de NFRs, métodos a serem aplicados e correlação) utilizando o NFR Framework para identificar as possíveis soluções a serem tratadas nas camadas da arquitetura MVC, de acordo com o atributo de qualidade de segurança;
	
	\item Identificar e selecionar as ferramentas de modelagem intencional que suportam o NFR Framework;
	
	\item Pesquisar e compreender como é o comportamento e a comunicação entre as camadas do padrão arquitetural MVC;
	
	\item Realizar modelagem do fluxo de atividades realizado para a aplicação da abordagem;
\end{itemize}

\chapter{Metodologia}

