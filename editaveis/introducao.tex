\chapter{Introdução}
\label{chap:introducao}

Neste capítulo, são descritos o contexto, no qual o trabalho está inserido; a questão de pesquisa, a qual norteia o embasamento científico do trabalho;  a justificativa, visando destacar a contribuição dessa pesquisa; os objetivos geral e específicos, com os quais são acordadas as principais metas a serem atingidas para conclusão do trabalho, e por fim, a organização dos demais capítulos dessa monografia.

\section{Contextualização}

Os requisitos não-funcionais (RNFs) e funcionais descrevem as características de um software, focando nas questões: \textbf{como esse software deve fazer?} e \textbf{o que esse software deve fazer?}  \cite{sommerville1997requirements}. Frequentemente, os RNFs são especificados de forma equivocada ou ainda são negligenciados, não sendo especificados \cite{eckhardt2016non}. Visando auxiliar os Engenheiros de Requisitos na tarefa de especificar os RNFs, os autores, em \cite{mairiza2010investigation}, destacam 114 classes de RNFs. Dentre essas classes, as mais mencionadas na literatura são: 1º Desempenho, 2º Confiabilidade e  3º Usabilidade. No contexto empresarial, ou seja, mais aplicado e comercial, as classes de RNFs mais utilizadas nas especificações são: 1º Segurança, 2º Confiabilidade e 3º Usabilidade \cite{eckhardt2016non}. Como o foco desse trabalho é atuar especificamente em suporte aplicáveis no contexto empresarial, pretende-se contribuir com a especificação do RNF de segurança, dado que é a classe de RNF mais relevante para o mercado, de acordo com a literatura investigada.

Ao desenvolver um software, os RNFs impactam, segundo \cite{eckhardt2016non}, na implementação, manutenção, operacionalização e utilização de recursos. Adicionalmente, impactam em aspectos arquiteturais de uma aplicação de software, sendo necessário especificar a arquitetura orientando-se não apenas pelos requisitos funcionais, mas também pelos RNFs \cite{buschmann1996system}

Dependendo do padrão arquitetural utilizado, as especificações bem como as operacionalizações dos RNFs podem variar \cite{chung2012non}. Diante do exposto, e visando focar em um padrão arquitetural comumente utilizado, o presente trabalho pretende orientar-se pelo padrão arquitetural \textit{Model-View-Controller} (MVC). Trata-se de um padrão bem aceito, bem como utilizado no desenvolvimento de aplicações Web, sendo inclusive base para o desenvolvimento dessas aplicações em \textit{frameworks} e plataformas de geração de código mais emergentes, orientadas à Convenção sobre Configuração \cite{jailia2016behavior}.


O MVC é um padrão arquitetural para sistemas interativos, que divide a aplicação em componentes: a \textit{Model}, que contém a implementação das funcionalidades principais e os dados da aplicação, ou seja as entidades de domínio; a \textit{View}, que exibe as informações da aplicação ao usuário, sendo, portanto, a camada mais próxima desse último, e a \textit{Controller}, que lida com as entradas do usuário \cite{buschmann1996system}, sendo um componente intermediário entre \textit{Model} e \textit{View}.  

Existem alguns \textit{frameworks} conceituais centrados na especificação de RNFs. Dentre eles, destacam-se: NFR Framework \cite{chung2009non}, i* \cite{istarwiki20} e FURPS \cite{umar2011analyzing}. Detalhes sobre esses frameworks serão cobertos no Capítulo \ref{chap:referencialTeorico} dessa monografia, Referencial Teórico. Em um primeiro momento, pode-se destacar que os dois primeiros, NFR Framework e i*, fazem uso de uma abordagem mais emergente para especificação desses RNFs. Já o FURPS pode ser entendido como um documento simples, especificado em linguagem natural, que se orienta por alguns RNFs comumente encontrados na literatura, no caso (em inglês): \textit{Functionality}, \textit{Usability}, \textit{Reliability}, \textit{Performance} e \textit{Supportability}.

A abordagem mais emergente, mencionada para os \textit{frameworks} NFR e i*, usa modelos específicos, focados na especificação de requisitos usando princípios da \textit{Goal-Oriented Requirements Engineering} (GORE) \cite{horkoff2016goal}. Rastreabilidade de requisitos \cite{wiegers2013software}; especificação de impactos e interdependências entre os requisitos; registro de alternativas quanto aos requisitos analisados, e definição de operacionalizações para viabilizar a realização dos requisitos são algumas vantagens que ficam evidentes no uso desses modelos mais específicos, como, por exemplo: o \textit{Softgoal Interdependency Graphs} (SIGs) \cite{chung2012non} ou, em português, Grafos de Interdependências entre Requisitos Não Funcionais. 

Portanto, esse trabalho propôs a definição de um Catálogo de SIGs centrado no RNF Segurança, especificamente desenhado para o Padrão Arquitetural MVC, usando a notação do NFR Framework. Com a intenção de auxiliar os Engenheiros de Requisitos e ainda os Engenheiros de Software na tarefa de especificação de requisitos não funcionais e funcionais quando uma arquitetura for imposta. Como o catálogo ficaria muito abrangente e generalista, caso fosse definido para qualquer RNF e padrão arquitetural, não sendo, portanto, útil, focou-se: (i) no RNF mais preocupante - Segurança, segundo a literatura - em aplicações de software de cunho comercial; (ii) em aplicações Web, e (iii) no padrão arquitetural MVC.  Vale ressaltar ainda que os termos RNFs, atributos de qualidade, critérios de qualidade e metas flexíveis serão utilizados como sinônimos ao longo dessa monografia.


\section{Questão de Pesquisa}
Este trabalho procurou responder a seguinte questão de pesquisa: \textit{Como auxiliar Engenheiros de Requisitos e Engenheiros de Software na especificação de RNFs quando uma arquitetura é imposta?} É relevante comentar que essa resposta ao final do trabalho, pareceu parcial, pois o trabalho exemplificou esse auxílio usando como base o RNF Segurança, aplicações Web e o padrão arquitetural MVC. Entretanto, cabe ao  ao leitor extrapolar as particularidades desse exemplo, usando-o como base mesmo quando outro RNF for foco bem como outro padrão arquitetural for imposto.

É recomendado que uma questão de pesquisa seja organizada em três visões \cite{keele2007guidelines}: 
\textbf{População}, estabelecendo quem são as pessoas diretamente afetadas pela intervenção; \textbf{Intervenção}, comparando dois ou mais “tratamentos alternativos”, ou seja, duas ou mais possíveis estratégias de atuação, e \textbf{Saídas}, estabelecendo a forma como a intervenção vem sendo aplicada. Nesse contexto, tem-se: 

\begin{itemize}
	\item População: “\textit{..Engenheiros de Requisitos e Engenheiros de Software na especificação de requisitos associados ao RNF Segurança, em aplicações Web desenvolvidas orientando-se pelo padrão arquitetural MVC..}”
	\item Intervenção: “\textit{..utilização de notação específica, NFR Framework, para modelagem dos Grafos de Interdependências de Requisitos Não Funcionais centrados no critério de qualidade Segurança, especificando impactos, interdependências e operacionalizações cabíveis  ..}”
	\item Saídas: “\textit{.. catálogo de soluções centrado em Segurança e no padrão arquitetural MVC..}”
\end{itemize}
\section{Justificativa}


Observou-se que mesmo com uma \textit{baseline}  arquitetural definida, os RNFs, chamados também de atributos ou critérios de qualidade, são negligenciados ou tratados de forma equivocada \cite{eckhardt2016non}. Entretanto, esses critérios evidenciam preocupações, as quais impactam na qualidade do software em desenvolvimento \cite{schneidewind1990standard}.  
Portanto, são necessários esforços para auxiliar os Engenheiros de Requisitos e os Engenheiros de Software na especificação desses RNFs, em particular em aplicações tipicamente comerciais, cujas arquiteturas são, normalmente, impostas pelas empresas de desenvolvimento.

Nesse trabalho então definiu um catálogo de SIGs centrado no RNF Segurança, no escopo das aplicações web, e no padrão arquitetural MVC aplicado em cenários de desenvolvimento onde a tecnologia utilizada foi \textit{rails} que trata-se de um framework de desenvolvimento web baseado no padrão arquitetural MVC. 

\section{Objetivos}
 
\subsection{Objetivo Geral}

Definir catálogo de segurança para o padrão arquitetural MVC usando modelagem orientada a grafos de interdependências de RNFs, ou seja, levando em consideração as boas práticas acordadas na Engenharia de Requisitos Orientada à Meta (em inglês, \textit{Goal Oriented Requirements Engineering} - GORE). 

\subsection{Objetivos Específicos}

Com intuito de alcançar o objetivo geral, foram considerados relevantes os seguintes objetivos específicos:

\begin{itemize}
	
	\item Investigar - na literatura: (i) formas recomendadas para lidar com o RNF Segurança em aplicações web desenvolvidas com base no padrão arquitetural MVC, permitindo identificar alternativas e operacionalizações para concretização desse RNF de forma satisfatória, e (ii) RNFs associados à Segurança, permitindo identificar as interdependências e os impactos entre eles;
	
	\item Elaborar o catálogo de SIGs, com base nos resultados da etapa de investigação (supracitada) bem como orientando-se pela notação do NFR Framework;
	
	\item Realizar a correspondência entre o  catálogo e as camadas do padrão arquitetural MVC;
	
	
	\item Elaborar cenários e desenvolver aplicações web exemplo, no padrão MVC, orientando-se pelo catálogo. Essas aplicações web exemplo podem ser vistas como cenários de uso ou um estudo de caso, procurando apresentar aos interessados como o catálogo pode ser utilizado/instanciado. 
	
\end{itemize}

\section{Organização da Monografia}

Esta monografia está dividida nos respectivos capítulos, sendo eles: 
\begin{itemize}
	\item O Capítulo \ref{chap:referencialTeorico} \textbf{Referencial Teórico}, explora os conceitos chave para elaboração da monografia, partindo da engenharia de requisitos e seus conceitos, explora-se também os aspectos e conceitos de arquitetura de software e o padrão MVC, bem como a visão dos requisitos de segurança como RNFs;
	
	\item  O Capítulo \ref{chap:suporteTecnologico} \textbf{Suporte Tecnológico}, apresenta as ferramentas de modelagem dos diagramas e mapas mentais, e as ferramentas de suporte à escrita utilizadas para o desenvolvimento da monografia;
	
	\item O Capítulo \ref{chap:metodologia} \textbf{Metodologia}, enfoca as questões da classificação da pesquisa e os procedimentos metodológicos para o desenvolvimento dessa monografia;
	
	\item O Capítulo \ref{chap:proposta} \textbf{Proposta}, descreve o catálogo de tipos de RNFs para segurança e o gráfico de interdependência entre as metas flexíveis (SIGs);
	
	\item  O Capítulo \ref{chap:consideracoesFinais} \textbf{Considerações Finais}, apresenta os resultados alcançados com a execução do TCC1 e o que espera alcançar com a execução do TCC2.
\end{itemize}

