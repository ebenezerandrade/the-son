\chapter{Suporte Tecnológico}
\label{chap:suporteTecnologico}

Neste Capítulo, são apresentadas as ferramentas de software utilizadas para dar suporte ao desenvolvimento deste trabalho. A seção \ref{sec:ferramentasModelagem} apresenta uma breve descrição das ferramentas utilizadas na modelagem dos diagramas e mapas mentais. A seção \ref{sec:ferramentasDesenvolvimento} apresenta as ferramentas de suporte à escrita, e para a realização do controle de versão. A seção \ref{sec:ferramentasParaDesenvolvimentoWebApp} apresenta as ferramentas que serão utilizadas para o desenvolvimento da aplicação web durante a execução do TCC2.  

\section{Ferramentas de Modelagem}
\label{sec:ferramentasModelagem}

\begin{itemize}
	\item \textbf{Astah Professional}: é uma ferramenta de modelagem de diagramas dinâmicos e estáticos, que suporta a UML 2.x, \textit{Entity Relationship Diagram} (ERD), \textit{Data Flow Diagram} (DFD), fluxogramas, mapas mentais, e a engenharia reversa para as linguagens Java, C\# e C++ \cite{astah}. A versão utilizada foi a v7.2.0 com licença de estudante, para modelagem dos diagramas da UML e mapas mentais.   
	
	\item \textbf{StarUML}: É um software \textit{open source} para modelagem de diagramas, na notação UML/\textit{Model Driven Architecture} (MDA). A proposta da ferramenta é tornar-se uma ferramenta gratuita que substitua as plataformas comerciais \cite{starUML}. A versão utilizada é a versão v1.0, pois suporta o \textit{plug-in} RE-Tools, o qual permite a modelagem do SIGs para o NFR Framework. 
	
	\item \textbf{OpenOME}: É uma ferramenta \textit{open source} de modelagem para apoiar a Engenharia Requisitos Orientada à Meta, orientada a agente e orientada a aspectos. Promove ao desenvolvedor um vínculo entre os requisitos e as especificações e as fases do \textit{design} arquitetural \cite{openOME}. A versão utilizada é a v3.4.1, para modelagem na notação do i*. 
	
	\item \textbf{Xmind}: É um software proprietário para criação de mapas mentais e suporte na criação de \textit{brainstorming} \cite{xMind}. A versão utilizada é o XMind 8 Update 2, permitindo a criação de mapas mentais. 
	
	\item \textbf{Bizagi Modeler}: É uma ferramenta gratuita utilizada para modelagem de processos de negócio, utilizando notação BPMN. É eleita pela comunidade a mais potente e fácil de utilizar do mercado. A versão utilizada é a versão 3.1, para a modelagem do fluxo de desenvolvimento da monografia \cite{bizagi}. 
	
	\item \textbf{RE-Tools}: É um conjunto de ferramentas \textit{open source} utilizado para a modelagem de diferentes aspectos das organizações e de seus sistemas, durante a realização das atividades da engenharia de requisitos, possuindo suporte para as notações do (i) NFR Framework, (ii) i*, (iii) \textit{Knowledge Acquisition in autOmated Specification} (KAOS), (iv) \textit{Problem Frames}, (v) UML e (vi) BPMN \cite{reTools} \cite{supakkul2012re}.
	
	Esse conjunto de ferramentas vem sendo utilizado para o ensino da Engenharia de Requisitos em cursos na Universidade de Trento (Itália), Universidade do Texas em Dallas (Estados Unidos) e na Universidade de Brasília (Brasil), além de estar envolvida em pesquisas no Brasil, Austrália, Canadá, China, França, Reino Unido e Estados Unidos \cite{supakkul2012re}. 
	
	No presente trabalho a versão do \textit{plug-in} utilizada foi a v3.0.2, para modelagem do gráfico de SIGs. 
\end{itemize}

\section{Ferramentas para Desenvolvimento da Monografia}
\label{sec:ferramentasDesenvolvimento}

\begin{itemize}
	
	\item \textbf{Git}: É um sistema \textit{open source} que suporta o controle de versão distribuído \cite{git}. A versão utilizada é a v2.13.0 para realizar o controle de versão da parte escrita da monografia. 
	
	\item \textbf{GitHub}: O GitHub é uma plataforma de hospedagem de código para controle de versão utilizando o Git, através do GitHub é possível criar uma conta para armazenar o código fonte de projetos em repositórios gratuitos e privados \cite{github}. Utilizado para hospedar o repositório público para o controle de versão da monografia, a versão utilizada é a v1.0.13.
	
	\item \LaTeX\ : É um sistema que permite a elaboração de textos em alta qualidade, através do programa de diagramação de textos TEX \cite{latex}. A versão utilizada é a v5.6.2 utilizado no desenvolvimento da parte escrita da monografia. 
	
\end{itemize}

\section{Ferramentas para o Desenvolvimento da Aplicação Web}
\label{sec:ferramentasParaDesenvolvimentoWebApp}

O Git e o GitHub também serão utilizados para realização do controle de versão de código no desenvolvimento do software.

\begin{itemize}
	
	\item \textbf{Linux Mint 18.3 Sylvia}: É uma distribuição linux baseada no Ubuntu 16.04 LTS é a terceira e última versão do ciclo da série 18 \cite{Mint}. Essa versão será utilizada no sistema operacional para o desenvolvimento do TCC2, a versão encontra-se disponível através da URL \url{https://linuxmint.com/download.php} 
	
	\item \textbf{Sublime Text:} É um editor de textos multiplataforma. Possuindo recursos que facilitam a escrita de código, como: Minimap, edição de textos em multi-painel, salvamento automático, autocompletar, correspondência entre parenteses, dentre outros \cite{SublimeText}. A versão que será utilizada no TCC2 é a versão 3.0.
	
	\item \textbf{Ruby:} É uma linguagem de programação interpretada e multiparadigma, desenvolvida por Yukihiro “Matz” Matsumoto, unificando caraterísticas de suas linguagens favoritas (Perl, Smalltalk, Eiffel, Ada e Lisp), para formar uma linguagem que equilibra a programação funcional com a programação imperativa \cite{Ruby}. A versão da linguagem que será utilizada no TCC2 será a versão 2.5.0.
	
	\item \textbf{Ruby on Rails - (RoR):} É um  \textit{framework open source} de desenvolvimento de aplicações web, escrito em \textbf{Ruby} e baseado no padrão arquitetural MVC \cite{RoR}. A versão do \textit{framework} que será utilizada no TCC2 será a versão 5.1.5.
	
	\item \textbf{MySQL}: De acordo com a Oracle o MySQL é o Sistema de Gerenciamento de Banco de Dados (SGBD) \textit{open source} mais conhecido do mundo, além de ser uma das principais tecnologias de SGBD para desenvolvimento de aplicações web \cite{MySQL}. A versão do MySQL quer será utilizada no TCC2 será o MySQL \textit{Community Server}, sob licença GPL disponível na versão 5.7.21.
	
	\item \textbf{Bootstrap:} É um \textit{framework open source} de desenvolvimento dos componentes responsivos para interface gráfica de aplicações web, que utiliza as Linguagens HTML, CSS e JavaScript \cite{Bootstrap}. A versão que será utilizada no TCC2 será a versão 4.0.0. 
	
\end{itemize}

\section{Resumo do Capítulo}

Neste capítulo, foram descritas as principais ferramentas que apoiaram o desenvolvimento do trabalho. Na seção \ref{sec:ferramentasModelagem} foi apresentada as versões e a utilização de cada ferramenta; a seção \ref{sec:ferramentasDesenvolvimento} apresenta as ferramentas utilizadas no desenvolvimento da parte escrita e o controle de versão da mesma, e a seção \ref{sec:ferramentasParaDesenvolvimentoWebApp} apresenta as ferramentas que serão utilizadas no desenvolvimento da aplicação web em tempo de execução do TCC2. 

A tabela \ref{resumo-cap-3}, resume as principais ferramentas que foram e que serão utilizadas em tempo de execução de TCC1 e TCC2. 

\begin{table}[]
	\centering
	\caption{Resumo das ferramentas de suporte.}
	\label{resumo-cap-3}
	\begin{tabular}{@{}cccc@{}}
		\toprule
		\textbf{Ferramenta} & \textbf{Versão} & \textbf{Suporte} & \textbf{Aplicabilidade} \\ \midrule
		\begin{tabular}[c]{@{}c@{}}Astah \\ Professional\end{tabular} & v7.2.0 & Modelagem & Modelagem de diagramas dinâmicos e estáticos. \\
		\rowcolor[HTML]{C0C0C0} 
		StarUML & v1.0 & Modelagem & Modelagem de diagramas na notação UML/MDA. \\
		OpenOME & v3.4.1 & Modelagem & Modelagem na notação do i*. \\
		\rowcolor[HTML]{C0C0C0} 
		Xmind & \begin{tabular}[c]{@{}c@{}}XMind 8 \\ Update 2\end{tabular} & Modelagem & Modelagem de mapas mentais. \\
		\begin{tabular}[c]{@{}c@{}}Bizagi \\ Modeler\end{tabular} & v3.1 & Modelagem & \begin{tabular}[c]{@{}c@{}}Modelagem de processos de negócio, \\ utilizando notação BPMN.\end{tabular} \\
		\rowcolor[HTML]{C0C0C0} 
		RE-Tools & v3.0.2 & Modelagem & Modelagem do gráfico de SIGs. \\
		Git & v2.13.0 & \begin{tabular}[c]{@{}c@{}}Controle de\\ versão\end{tabular} & \begin{tabular}[c]{@{}c@{}}Realização do controle de versão da escrita e \\ para o desenvolvimento da aplicação web em \\ tempo de execução do TCC2.\end{tabular} \\
		\rowcolor[HTML]{C0C0C0} 
		GitHub & v1.0.13 & \begin{tabular}[c]{@{}c@{}}Hospedagem\\ de código fonte\end{tabular} & \begin{tabular}[c]{@{}c@{}}Armazenar o código fonte da parte escrita do \\ trabalho e da aplicação web que será desenvolvida\end{tabular} \\
		LaTeX & v5.6.2 & \begin{tabular}[c]{@{}c@{}}Escrita/\\ formatação\end{tabular} & Utilizado para a escrita e formatação do texto. \\
		\rowcolor[HTML]{C0C0C0} 
		Linux Mint & \begin{tabular}[c]{@{}c@{}}v18.3 \\ Sylvia\end{tabular} & \begin{tabular}[c]{@{}c@{}}Sistema \\ Operacional\end{tabular} & \begin{tabular}[c]{@{}c@{}}Será o sistema operacional para o desenvolvimento\\ da aplicação web.\end{tabular} \\
		Sublime Text & v3.0 & \begin{tabular}[c]{@{}c@{}}Escrita de \\ código\end{tabular} & \begin{tabular}[c]{@{}c@{}}Editor de textos para escrita do código fonte \\ da aplicação web.\end{tabular} \\
		\rowcolor[HTML]{C0C0C0} 
		Ruby & v2.5.0 & \begin{tabular}[c]{@{}c@{}}Linguagem \\ de programação\end{tabular} & \begin{tabular}[c]{@{}c@{}}Linguagem de programação para desenvolvimento\\ da aplicação web.\end{tabular} \\
		RoR & v5.1.5 & Desenvolvimento & \begin{tabular}[c]{@{}c@{}}Framework de desenvolvimento web de acordo\\ com o padrão arquitetural MVC.\end{tabular} \\
		\rowcolor[HTML]{C0C0C0} 
		MySQL & v5.7.21 & Desenvolvimento & \begin{tabular}[c]{@{}c@{}}SGBD que será utilizado no desenvolvimento da \\ aplicação web.\end{tabular} \\
		Bootstrap & v4.0.0 & Desenvolvimento & \begin{tabular}[c]{@{}c@{}}Framework quer será utilizado para desenvolver\\ o front-end da aplicação web.\end{tabular} \\ \bottomrule
	\end{tabular}
\end{table}
