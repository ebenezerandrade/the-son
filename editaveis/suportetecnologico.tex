\chapter{Suporte Tecnológico}
\label{chap:suporteTecnologico}

Neste Capítulo, são apresentadas as ferramentas de software que apoiaram o desenvolvimento desse trabalho. A seção \ref{sec:ferramentasModelagem} apresenta uma breve descrição das ferramentas utilizadas na modelagem dos diagramas e mapas mentais. A seção \ref{sec:ferramentasDesenvolvimento} apresenta as ferramentas de apoio à escrita, e para a realização do controle de versão. A seção \ref{sec:ferramentasParaDesenvolvimentoWebApp} apresenta as ferramentas utilizadas no desenvolvimento de diferentes cenários de uso para aplicação e evolução do Catálogo de Segurança.   

\section{Ferramentas de Modelagem}
\label{sec:ferramentasModelagem}

Parte das contribuições desse trabalho é dedicada à elaboração do Catálogo de Segurança, o qual é apoiado no referencial teórico de GORE, conforme acordado na seção \ref{subsec:orientacaoMeta}. Visando a representação desse catálogo em diferentes níveis de abstração, foram utilizadas algumas notações, no caso, UML, NFR \textit{Framework}, Mapa Mental \cite{xMind} e BPMN \cite{white2004introduction}.
 
A UML foi utilizada para especificação de alguns detalhes atrelados, principalmente, ao uso do Padrão Arquitetural MVC, evidenciando as correlações entre suas camadas, conforme consta na seção \ref{subsubsec:fluxoDeInteracaoEntreOsComponentes}. O NFR \textit{Framework} foi utilizado como base para especificação do Catálogo de Segurança como um todo, fazendo-se uso das abstrações de metas flexíveis, operacionalizações, alegações, dentre outras mais específicas dessa notação. Mapas mentais permitiram documentar \textit{brainstorming}, o qual representou uma técnica de elicitação muito utilizada ao longo desse trabalho. Por fim, para ilustrar as típicas atividades que conduziram a realização desse trabalho, optou-se por representar os fluxos de atividades em BPMN. Diante do exposto, foram necessários suportes tecnológicos que apoiassem as modelagens supracitadas, nas diferentes notações, conforme acordado a seguir:


\begin{itemize}
	\item \textbf{Astah Professional}: é uma ferramenta de modelagem de diagramas dinâmicos e estáticos, que se orienta pelas notações UML 2.x, \textit{Entity Relationship Diagram} (ERD), \textit{Data Flow Diagram} (DFD), fluxogramas, mapas mentais, e apoia a Engenharia Reversa para as linguagens Java, C\# e C++ \cite{astah}. A versão utilizada foi a v7.2.0, com licença de estudante, para modelagem dos diagramas da UML e mapas mentais;   
	
	\item \textbf{StarUML}: é um software \textit{open source} para modelagem de diagramas na notação UML/\textit{Model Driven Architecture} (MDA). A proposta da ferramenta é tornar-se uma ferramenta gratuita que substitua as plataformas comerciais \cite{starUML}. A versão utilizada foi a versão v1.0, pois é compatível com o\textit{plug-in} RE-\textit{Tools}, o qual permite a modelagem do SIGs para o NFR Framework;
	
	\begin{comment}
		\item \textbf{OpenOME}: É uma ferramenta \textit{open source} de modelagem para apoiar a Engenharia Requisitos Orientada à Meta, orientada a agente e orientada a aspectos. Promove ao desenvolvedor um vínculo entre os requisitos e as especificações e as fases do \textit{design} arquitetural \cite{openOME}. A versão utilizada foi a v3.4.1, para modelagem na notação do i*.
	\end{comment}
	
	\item \textbf{Xmind}: é um software proprietário para criação de mapas mentais e documentação de \textit{brainstorming} \cite{xMind}. A versão utilizada foi o XMind 8 Update 2, permitindo a criação de mapas mentais; 
	
	\item \textbf{Bizagi Modeler}: é uma ferramenta gratuita utilizada para modelagem de processos de negócio, utilizando a notação BPMN. A versão utilizada foi a versão 3.1, para a modelagem do fluxo de atividades que conduziram o desenvolvimento desse trabalho \cite{bizagi}, e 
	
	\item \textbf{RE-\textit{Tools}}: é um conjunto de ferramentas \textit{open source} utilizado para a modelagem de diferentes aspectos das organizações e de seus sistemas, durante a realização das atividades da Engenharia de Requisitos, conferindo apoio às notações do (i) NFR \textit{Framework}, (ii) i* \cite{istarwiki20}, (iii) \textit{Knowledge Acquisition in autOmated Specification} (KAOS) \cite{van2001goal}, (iv) \textit{Problem Frames} \cite{jackson2005problem}, (v) UML e (vi) BPMN \cite{reTools} \cite{supakkul2012re}.
	
	Esse conjunto de ferramentas vem sendo utilizado para o ensino da Engenharia de Requisitos em cursos na Universidade de Trento (Itália), Universidade do Texas em Dallas (Estados Unidos) e na Universidade de Brasília (Brasil). Além disso, essas ferramentas constam em pesquisas realizadas em: Brasil, Austrália, Canadá, China, França, Reino Unido e Estados Unidos \cite{supakkul2012re}.
	
	No presente trabalho, a versão do \textit{plug-in} utilizada foi a v3.0.2, para modelagem do gráfico de SIGs. 
\end{itemize}

\section{Ferramentas para Desenvolvimento do Trabalho}
\label{sec:ferramentasDesenvolvimento}

A escrita da monografia também demandou suportes tecnológicos específicos, sendo os principais:

\begin{itemize}
	
	\item \textbf{Git}: é um sistema \textit{open source} que permite manter o controle de versão distribuído \cite{git}. A versão utilizada foi a v2.13.0, para realizar, o controle de versão da parte escrita do trabalho;
	
	\item \textbf{GitHub}: o GitHub é uma plataforma de hospedagem de código para controle de versão utilizando o Git, através do GitHub, é possível criar uma conta para armazenar o código fonte de projetos em repositórios gratuitos e privados \cite{github}. Utilizado para hospedar o repositório público para o controle de versão da monografia, sendo a versão utilizada a v1.0.13, e
	
	\item \LaTeX\ : é um sistema que permite a elaboração de textos em alta qualidade, através do programa de diagramação de textos TEX \cite{latex}. A versão utilizada foi a v5.6.2, para documentação da parte escrita.
	
\end{itemize}

\section{Ferramentas para o Desenvolvimento da Aplicação Web}
\label{sec:ferramentasParaDesenvolvimentoWebApp}

Por fim, constam os suportes tecnológicos que viabilizaram o desenvolvimento dos cenários de uso, nos quais o Catálogo de Segurança foi aplicado. Isso permitiu evoluir o catálogo, obtendo versões do mesmo cada vez mais refinadas. Cabe ressaltar que, o Git e o GitHub também foram utilizados para controle das versões de código, oriundas dos produtos de software de cada cenário de uso. Seguem esses suportes:

\begin{itemize}
	
	\item \textbf{Linux Mint 18.3 Sylvia}: é uma distribuição linux, baseada no Ubuntu 16.04 LTS, sendo a terceira e última versão do ciclo da série 18 \cite{Mint}. Essa versão do sistema operacional foi utilizada em atendimento ao ambiente de desenvolvimento dos cenário, A versão utilizada encontra-se disponível em  \url{https://linuxmint.com/download.php};
	
	\item \textbf{Sublime Text:} é um editor de textos multiplataforma. Esse editor possui recursos que facilitam a escrita de código, como: Minimap, edição de textos em multi-painel, salvamento automático, autocompletar, correspondência entre parenteses, dentre outros \cite{SublimeText}. A versão utilizada foi a versão 3.0;
	
	\item \textbf{Ruby:} é uma linguagem de programação interpretada e multiparadigma, desenvolvida por Yukihiro “Matz” Matsumoto, unificando caraterísticas de suas linguagens favoritas (Perl, Smalltalk, Eiffel, Ada e Lisp), para formar uma linguagem que equilibra a programação funcional com a programação imperativa \cite{Ruby}. A versão da linguagem utilizada foi a versão 2.5.0;
	
	\item \textbf{Ruby on Rails - (RoR):} é um  \textit{framework open source} de desenvolvimento de aplicações web, escrito em \textbf{Ruby}, e baseado no Padrão Arquitetural MVC \cite{RoR}. A versão do \textit{framework} utilizada foi versão 5.1.5;
	
	\item \textbf{MySQL}: de acordo com a Oracle, o MySQL é o Sistema de Gerenciamento de Banco de Dados (SGBD) \textit{open source} mais conhecido do mundo, além de ser uma das principais tecnologias de SGBD para desenvolvimento de aplicações web \cite{MySQL}. A versão do MySQL utilizada foi o MySQL \textit{Community Server}, sob licença GPL, disponível na versão 5.7.21, e
	
	\item \textbf{Bootstrap:} é um \textit{framework open source} de desenvolvimento dos componentes responsivos para interface gráfica de aplicações web, que utiliza as Linguagens HTML, CSS e JavaScript \cite{Bootstrap}. A versão utilizada foi a versão 4.0.0. 
	
\end{itemize}


\section{Resumo do Capítulo}

Neste capítulo, foram descritas as principais ferramentas que apoiaram o desenvolvimento do trabalho. Na seção \ref{sec:ferramentasModelagem} foram apresentadas as versões e a utilização de cada ferramenta; a seção \ref{sec:ferramentasDesenvolvimento} descreveu as ferramentas utilizadas no desenvolvimento da parte escrita e o controle de versão da mesma, e a seção \ref{sec:ferramentasParaDesenvolvimentoWebApp} acordou as ferramentas que foram utilizadas no desenvolvimento dos cenários de aplicação do catálogo. 


A tabela \ref{resumo-cap-3} resume as principais ferramentas que foram utilizadas no trabalho.

\begin{table}[h!]
	\centering
	\caption{Resumo das ferramentas de apoio.}
	\label{resumo-cap-3}
	\begin{tabular}{@{}cccc@{}}
		\toprule
		\textbf{Ferramenta} & \textbf{Versão} & \textbf{Apoio} & \textbf{Aplicabilidade} \\ \midrule
		\begin{tabular}[c]{@{}c@{}}Astah \\ Professional\end{tabular} & v7.2.0 & Modelagem & Modelagem de diagramas dinâmicos e estáticos. \\
		\rowcolor[HTML]{C0C0C0} 
		StarUML & v1.0 & Modelagem & Modelagem de diagramas na notação UML/MDA. \\ 
		Xmind & \begin{tabular}[c]{@{}c@{}}XMind 8 \\ Update 2\end{tabular} & Modelagem & Modelagem de mapas mentais. \\
		\rowcolor[HTML]{C0C0C0}
		\begin{tabular}[c]{@{}c@{}}Bizagi \\ Modeler\end{tabular} & v3.1 & Modelagem & \begin{tabular}[c]{@{}c@{}}Modelagem de processos de negócio, \\ utilizando a notação BPMN.\end{tabular} \\ 
		RE-Tools & v3.0.2 & Modelagem & Modelagem do gráfico de SIGs. \\
		\rowcolor[HTML]{C0C0C0} Git & v2.13.0 & \begin{tabular}[c]{@{}c@{}}Controle de\\ versão\end{tabular} & \begin{tabular}[c]{@{}c@{}}Realização do controle de versão da escrita e \\  dos códigos das aplicações nos cenários.\end{tabular} \\ 
		GitHub & v1.0.13 & \begin{tabular}[c]{@{}c@{}}Hospedagem\\ de código fonte\end{tabular} & \begin{tabular}[c]{@{}c@{}}Armazenamento de código fonte da parte escrita do \\ trabalho e das aplicação web desenvolvidas \\ com base nos cenários de uso estabelecidos.\end{tabular} \\
		\rowcolor[HTML]{C0C0C0}
		LaTeX & v5.6.2 & \begin{tabular}[c]{@{}c@{}}Escrita/\\ formatação\end{tabular} & Escrita e formatação do texto. \\ 
		Linux Mint & \begin{tabular}[c]{@{}c@{}}v18.3 \\ Sylvia\end{tabular} & \begin{tabular}[c]{@{}c@{}}Sistema \\ Operacional\end{tabular} & \begin{tabular}[c]{@{}c@{}} Sistema operacional para o desenvolvimento\\ dos cenários de uso.\end{tabular} \\
		\rowcolor[HTML]{C0C0C0}
		Sublime Text & v3.0 & \begin{tabular}[c]{@{}c@{}}Escrita de \\ código\end{tabular} & \begin{tabular}[c]{@{}c@{}} Edição de textos para escrita do \\ código fonte das aplicações dos cenários de uso.\end{tabular} \\ 
		Ruby & v2.5.0 & \begin{tabular}[c]{@{}c@{}}Linguagem \\ de programação\end{tabular} & \begin{tabular}[c]{@{}c@{}}Linguagem de programação para desenvolvimento\\ das aplicações nos cenários de uso.\end{tabular} \\
		\rowcolor[HTML]{C0C0C0}
		RoR & v5.1.5 & Desenvolvimento & \begin{tabular}[c]{@{}c@{}}\textit{Framework} de desenvolvimento web de acordo\\ com o Padrão Arquitetural MVC.\end{tabular} \\
		MySQL & v5.7.21 & Desenvolvimento & \begin{tabular}[c]{@{}c@{}}SGBD para desenvolvimento das aplicações nos \\ cenários de uso.\end{tabular} \\
		\rowcolor[HTML]{C0C0C0} 
		Bootstrap & v4.0.0 & Desenvolvimento & \begin{tabular}[c]{@{}c@{}}\textit{Framework} para desenvolvimento das aplicações\\ no cenários de uso.\end{tabular} \\ \bottomrule
	\end{tabular}
\end{table}
