\chapter[Referencial Teórico]{Referencial Teórico}

\section{Engenharia de Requisitos Orientada a Meta}

\subsection{Requisitos Funcionais}

\subsection{Requisitos Não-Funcionais}

\section{Modelagem Intencional}

\subsection{NFR Framework}

O NFR Framework é um modelo intencional que trata os requisitos não-funcionais em grafos, que podem ser visualizados através da construção interativa e incremental, da análise e revisão de um \textit{Softgoal Interdependency Graph} - SIG. É através do SIG que surge os principais conceitos do NFR Framework. como \textit{softgoals} que são os principais requisitos e são apresentados como uma nuvem na parte superior do grafo, esses \textit{softgoals} podem ser conectados através de links de interdependência (que são desenhados como linhas) com \textit{softgoals} mais flexíveis\cite{chung2012non}.

Os \textit{softgoals} estão associados ao conceito de \textit{labels} - valores que representam o grau em que um \textit{sofgoal} pode ser associado. Os diferentes tipos de \textit{labels} são: Satisfeito, Fracamente satisfeito, Negado, fracamente negado, incediso e crítico \cite{chung2012non}. 

\section{Frameworks para Arquitetura Empresarial}