\begin{resumo}
 Este trabalho teve como principal objetivo a construção de um catálogo de segurança para o Padrão Arquitetural MVC (\textit{Model-View-Controller}), visando auxiliar Engenheiros de Requisitos e Engenheiros de Software na especificação dos requisitos associados ao requisito não funcional de Segurança. Utilizando modelagem orientada a grafos de interdependência de requisitos não funcionais, levando em consideração as boas práticas da Engenharia de Requisitos Orientada à Meta. Foi então, modelado na notação do NFR \textit{Framework} o Catálogo de Segurança, fundamentando-se principalmente nos conceitos de Segurança apresentados por Chung, baseando-se em Confidencialidade, Integridade e Disponibilidade, adicionalmente o catálogo apoia na definição de Segurança da Informação definida na ISO 27001. Além disso, foi aplicado ao padrão arquitetural MVC, onde verifica-se a relação das metas flexíveis e operacionalizações com as camadas do Padrão Arquitetural MVC.

 \vspace{\onelineskip}
    
 \noindent
 \textbf{Palavras-chaves}: NFR \textit{Framework}. \textit{Model-View-Controller}. Segurança de Software. Engenharia de Requisitos Orientada à Meta. Catálogo de Segurança. Padrão Arquitetural. \textit{Softgoal Interdependency Graphs}.
\end{resumo}
