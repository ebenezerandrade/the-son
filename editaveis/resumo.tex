\begin{resumo}
 Este trabalho teve como principal objetivo a construção de um catálogo de segurança para o Padrão Arquitetural MVC (\textit{Model-View-Controller}), visando auxiliar Engenheiros de Requisitos e Engenheiros de Software na especificação dos requisitos associados ao requisito não funcional Segurança. O catálogo encontra-se documentado com base em uma modelagem orientada a grafos de interdependência de requisitos não funcionais, levando em consideração boas práticas da Engenharia de Requisitos Orientada à Meta. Na construção do catálogo, foram considerados alguns pilares, sendo esses: conceitos associados à Segurança e fundamentados em autores da área, com ênfase em Confidencialidade, Integridade e Disponibilidade, e a definição de Segurança da Informação estabelecida pela ISO 27001. Como uma extensão das contribuições base do catálogo, os apontamentos do mesmo foram mapeados/associados às camadas do Padrão MVC, no intuito de auxiliar ainda mais os usuários do catálogo, enquanto desenvolvem suas aplicações baseadas nesse padrão arquitetural em três camadas. Por fim, tem-se a aplicação do catálogo em diferentes cenários de uso, o que possibilitou não apenas um catálogo em alto nível de abstração, baseado nas interdependências dos requisitos não funcionais correlacionados à Segurança, mas também um catálogo que acorda uma série de operacionalizações, as quais procuram cumprir (de forma satisfatória pelo menos) com os requisitos não funcionais especificados. 

 \vspace{\onelineskip}
    
 \noindent
 \textbf{Palavras-chaves}: Catálogo de Segurança de Software. Padrão Arquitetural \textit{Model-View-Controller}. Engenharia de Requisitos Orientada à Meta. Segurança. Confidencialidade. Integridade. Disponibilidade.
\end{resumo}
