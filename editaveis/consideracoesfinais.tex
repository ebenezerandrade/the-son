\chapter{Conclusão}
\label{chap:consideracoesFinais}

\begin{comment}


	
	
	Esse Capítulo procura apresentar um resumo quanto aos resultados alcançados até o momento, com a realização do presente trabalho, bem como o que ainda será alcançado com a realização do TCC2. Dessa forma, a seção \ref{sec:resultadosObtidos} procura resgatar os objetivos geral e específicos apresentados no Capítulo \ref{chap:introducao}, detalhando em uma tabela (ou em um quadro) o que foi atendido em tempo de TCC1. Na seção \ref{sec:resultadosEsperados}, outra tabela (ou outro quadro) é apresentada(o), procurando acordar o que ainda será atendido em tempo de TCC2.
	
	\section{Resultados Esperados}
	\label{sec:resultadosEsperados}
	
	É muito provável em tempo de execução do TCC2 o nível de mapeamento entre as metas flexíveis e o  Padrão Arquitetural MVC possa ser refinado. Acredita-se que ao desenvolver o software utilizando o Padrão arquitetural MVC, será possível compreender mais facilmente as relações entre as metas flexíveis e as operacionalizações com o padrão. 
	
	
	\begin{table}[h!]
	\centering
	\caption{Resultados esperados de acordo com os objetivos específicos.}
	\label{resultadosEsperados}
	\tiny
	\begin{tabular}{@{}p{8cm}p{7.5cm}@{}}
	\toprule
	\textbf{Objetivo} & \textbf{Resultado esperado} \\ \midrule
	Investigar na literatura formas de lidar com o RNF Segurança,em aplicações Web desenvolvidas utilizando o MVC. & Espera-se com a implementação do software validar as formas de lidar com o RNF de Segurança. \\
	\rowcolor[HTML]{C0C0C0} 
	Investigar na literatura os RNF associados a segurança e identificar o impacto e as interdepêndencias entre eles. & Espera-se com a implementação do software identificar os impactos e as interdepêndencias com os RNFs de Segurança de acordo com as metas flexiveis a serem definidas de acordo com o contexto em que o software será desenvolvido. \\
	Desenvolver aplicação web utilizando o Padrão Arquitetural MVC & Através do desenvolvimento da aplicação realizar a coleta das primeiras impressões da aplicação do catálogo, evoluir o catálogo e identificar novas metas flexíveis e operacionalizações para evoluir o catálogo. \\ \bottomrule
	\end{tabular}
	\end{table}
	
	\section{Resumo do Capítulo}
	
	Acredita-se que o principal intuito dessa pesquisa esteja sendo alcançado, uma vez que o catálogo de Segurança pode ser visto como um exemplo de como aproximar a especificação de algo tão abstrato, como são os RNFs ou critérios de qualidade, de insumos mais concretos - operacionalizações, os quais podem ser mais bem compreendidos bem como mais úteis aos membros da equipe de desenvolvimento. Lembrando que esses membros atuam em atividades que se encontram em níveis mais baixos de abstração - ou seja, mais próximos de código. A intenção é ajudá-los no cumprimento dessas especificações de requisitos não-funcionais, comumente negligenciados \cite{eckhardt2016non}.
	
\end{comment}

