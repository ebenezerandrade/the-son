\chapter[Conclusão]{Conclusão}
\label{chap:consideracoesFinais}



Dada a complexidade do tema abordado, em especial por procurar contribuir a partir de critérios de qualidade que são intrinsecamente abstratos e subjetivos, observou-se que o Catálogo de Segurança reflete uma visão preliminar sobre Segurança. Espera-se ainda, com o Catálogo de Segurança, que as contribuições desse trabalho possam ser compreendidas, ajustadas e aplicadas para e em outros cenários de uso. A ideia foi demonstrar como é possível tratar aspectos subjetivos, como os RNFs, permitindo satisfazê-los com a implementação de operacionalizações. Ou seja, demonstrar os elos de ligação entre um RNF e algo mais viável de ser concretizado em termos de arquitetura e código. Além disso, tudo fica anotado, utilizando uma notação em diferentes níveis de abstração, e deixando claro os impactos e as correlações entre esses níveis.

Uma vez evidenciadas as correlações entre os níveis de abstração, dos requisitos ao código, procurou-se conferir uma ligação entre algo muito abstrato (requisitos), com algo muito concreto (código). Esses níveis de abstração são comumente vistos como muito distantes, envolvendo e demandando - muitas vezes - equipes técnicas diferentes. Tal distanciamento, pode incorrer no não entendimento, ou não atendimento correto, ou ainda no próprio esquecimento de alguns desses aspectos subjetivos. Essa prática, conforme acordado no referencial teórico desse trabalho, o qual foi apoiado na literatura da área, pode levar - e recorrentemente leva - a insucessos nos projetos de software. 


A Tabela \ref{resultadosObtidos} apresenta, de acordo com os objetivos específicos do trabalho, os níveis de satisfação e os motivos pelos quais os mesmos foram atendidos. 


\begin{table}[h!]
	\centering
	\caption{Níveis de satisfação dos objetivos específicos.}
	\label{resultadosObtidos}
	\tiny
	\begin{tabular}{@{}p{6cm}p{3cm}p{6cm}@{}}
		\toprule
		\textbf{Objetivo} & \textbf{Nível de satisfação} & \textbf{Motivo} \\ \midrule
		Investigar na literatura formas de lidar com o RNF Segurança  em aplicações Web desenvolvidas utilizando o MVC. &  Atendido & Atendido com base em estudos realizados na área de interesse, com os quais referências foram levantadas e sustentam a necessidade de se considerar o RNF Segurança bem como seus impactos em aplicações Web, desenvolvidas com base no Padrão Arquitetural MVC. \\
		\rowcolor[HTML]{C0C0C0} 
		Investigar na literatura os RNF associados à segurança, e identificar o impacto e as interdependências entre eles. & Atendido &  Atendido pela existência de fontes que evidenciam a relação entre os RNFs de segurança em diferentes níveis de abstração. \\
		Elaborar SIG & Atendido & Atendido, pois tem-se a primeira versão do catálogo elaborado com sucesso. \\
		\rowcolor[HTML]{C0C0C0} 
		Realizar correspondência entre o Catálogo de Segurança e as camadas do Padrão Arquitetural MVC & Atendido & Atendido, pois o presente trabalho demonstrou a existência e os impactos das relações entre os RNFs que geram impacto na Segurança do Software em três níveis de abstração \\
		
		Elaborar cenários e desenvolver aplicações web exemplo, no padrão arquitetural MVC, orientando-se pelo Catálogo de Segurança & Atendido & Atendido, pois, para validação do Catálogo de Segurança, o mesmo foi aplicado em cinco cenários, que podem ser vistos como cenários de uso, em alguns casos, e estudo de casos, em outros. \\
		 \bottomrule
	\end{tabular}
\end{table}

\section*{Trabalhos Futuros}

Muitos trabalhos futuros podem ser desenhados a partir das contribuições e iniciativas desse trabalho. Dentre as ideias para projetos futuros, destacam-se:

Podem ser realizadas correlações desse catálogo, focado em Segurança, com outros catálogo, em diferentes critérios de qualidade. De acordo com o levantamento realizado nesse trabalho, têm-se que os primeiros critérios candidatos à investigação, dada à relevância dos mesmos na literatura, são Desempenho e Usabilidade. Anotar os impactos entre esses catálogos e o presente catálogo poderá conferir uma série de novas colocações. Por exemplo, é claro que se segurança for respeitada ao extremo, isso impactará em aspectos de usabilidade. Solicitar nome do pai, nome da mãe, digital e data de nascimento, conferem maior segurança em um caixa eletrônico, mas comprometem aspectos de usabilidade. Essas e outras conclusões poderiam ser mapeadas, acordadas, permitindo estudos mais aprofundados desses aspectos subjetivos no desenvolvimento de software.  

O Catálogo de Segurança, apresentado neste trabalho, também pode evoluir a níveis cada vez mais baixos de abstração, sendo capaz de conferir respostas a questionamentos bem específicos, como: “qual método de criptografia pode ser mais eficiente para uma determinada operacionalização e que está diretamente ligado à segurança do software?”, ou ainda ser expandido para indagações na área de segurança da informação como auditoria e controle, contestabilidade e responsabilização, autenticidade ou confiabilidade.  

Por fim, poderiam ser mencionados suportes computacionais que permitissem que o catálogo fosse disponibilizado para a comunidade interessada, conferindo - inclusive - mecanismos de compartilhamento de conhecimento e possibilidades de evoluções colaborativas do catálogo. 

\begin{comment}
	Prática essa que pode levar a muitos insucessos, tais como o Caso da Ambulância de Londres \cite{finkelstein1996comedy}.
	
	“”
\end{comment}
 


\begin{comment}


	
	
	Esse Capítulo procura apresentar um resumo quanto aos resultados alcançados até o momento, com a realização do presente trabalho, bem como o que ainda será alcançado com a realização do TCC2. Dessa forma, a seção \ref{sec:resultadosObtidos} procura resgatar os objetivos geral e específicos apresentados no Capítulo \ref{chap:introducao}, detalhando em uma tabela (ou em um quadro) o que foi atendido em tempo de TCC1. Na seção \ref{sec:resultadosEsperados}, outra tabela (ou outro quadro) é apresentada(o), procurando acordar o que ainda será atendido em tempo de TCC2.
	
	\section{Resultados Esperados}
	\label{sec:resultadosEsperados}
	
	É muito provável em tempo de execução do TCC2 o nível de mapeamento entre as metas flexíveis e o  Padrão Arquitetural MVC possa ser refinado. Acredita-se que ao desenvolver o software utilizando o Padrão arquitetural MVC, será possível compreender mais facilmente as relações entre as metas flexíveis e as operacionalizações com o padrão. 
	
	
	\begin{table}[h!]
	\centering
	\caption{Resultados esperados de acordo com os objetivos específicos.}
	\label{resultadosEsperados}
	\tiny
	\begin{tabular}{@{}p{8cm}p{7.5cm}@{}}
	\toprule
	\textbf{Objetivo} & \textbf{Resultado esperado} \\ \midrule
	Investigar na literatura formas de lidar com o RNF Segurança,em aplicações Web desenvolvidas utilizando o MVC. & Espera-se com a implementação do software validar as formas de lidar com o RNF de Segurança. \\
	\rowcolor[HTML]{C0C0C0} 
	Investigar na literatura os RNF associados a segurança e identificar o impacto e as interdepêndencias entre eles. & Espera-se com a implementação do software identificar os impactos e as interdepêndencias com os RNFs de Segurança de acordo com as metas flexiveis a serem definidas de acordo com o contexto em que o software será desenvolvido. \\
	Desenvolver aplicação web utilizando o Padrão Arquitetural MVC & Através do desenvolvimento da aplicação realizar a coleta das primeiras impressões da aplicação do catálogo, evoluir o catálogo e identificar novas metas flexíveis e operacionalizações para evoluir o catálogo. \\ \bottomrule
	\end{tabular}
	\end{table}
	
	\section{Resumo do Capítulo}
	
	Acredita-se que o principal intuito dessa pesquisa esteja sendo alcançado, uma vez que o catálogo de Segurança pode ser visto como um exemplo de como aproximar a especificação de algo tão abstrato, como são os RNFs ou critérios de qualidade, de insumos mais concretos - operacionalizações, os quais podem ser mais bem compreendidos bem como mais úteis aos membros da equipe de desenvolvimento. Lembrando que esses membros atuam em atividades que se encontram em níveis mais baixos de abstração - ou seja, mais próximos de código. A intenção é ajudá-los no cumprimento dessas especificações de requisitos não-funcionais, comumente negligenciados \cite{eckhardt2016non}.
		
\end{comment}

